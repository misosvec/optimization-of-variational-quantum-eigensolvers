\usepackage{amsmath}
\usepackage{amsthm}
\usepackage{amssymb}
\usepackage{amsfonts}
\usepackage{mathtools}
\usepackage{mathrsfs}
% \usepackage{commath} buggy library that broke the look of matrices
\usepackage{bm}

\usepackage{subcaption}
\usepackage{graphicx}
\usepackage{xcolor}
\usepackage{tikz}
\usetikzlibrary{arrows.meta, quotes, calc}
\usetikzlibrary{positioning}

\usepackage{url}
\usepackage[hidelinks,breaklinks]{hyperref}
\usepackage{cleveref}
\usepackage{emptypage}

\usepackage{enumitem}
\usepackage{listings}
\usepackage{algpseudocode}
\usepackage{algorithm}
\usepackage{tabularx} % allows to set full width of table
\usepackage{makecell}
\usepackage{array}

\lstset{frame=lines}

\theoremstyle{plain}
\newtheorem{theorem}{Theorem}
\newtheorem{lemma}{Lemma}
\newtheorem{corollary}{Corollary}
\newtheorem{proposition}{Proposition}
\newtheorem{conjecture}{Conjecture}
\newtheorem{criterion}{Criterion}
\newtheorem{assertion}{Assertion}

\theoremstyle{definition}
\newtheorem{definition}{Definition}
\newtheorem{condition}{Condition}
\newtheorem{problem}{Problem}
\newtheorem{example}{Example}
\newtheorem{exercise}{Exercise}
\newtheorem{question}{Question}
\newtheorem{axiom}{Axiom}
\newtheorem{property}{Property}
\newtheorem{assumption}{Assumption}
\newtheorem{hypothesis}{Hypothesis}
\newtheorem{optimization}{Optimization} \crefname{optimization}{opt.}{opts.}

\theoremstyle{remark}
\newtheorem{remark}{Remark}
\newtheorem{note}{Note}
\newtheorem{notation}{Notation}
\newtheorem{claim}{Claim}
\newtheorem{summary}{Summary}
\newtheorem{acknowledgment}{Acknowledgment}
\newtheorem{case}{Case}
\newtheorem{conclusion}{Conclusion}

\newcommand{\todo}[1]{\textcolor{red}{TODO {#1}}}
\newcommand{\ques}[1]{\textcolor{blue}{QUESTION: {#1}}}

% Add your packages and custom commands here.
\newcommand{\bra}[1]{\langle#1\rvert} % Bra
\newcommand{\ket}[1]{\lvert#1\rangle} % Ket
\newcommand{\braket}[2]{\langle#1\vert#2\rangle} % Bra-ket

% Exclude section from TOC but still keep numbering
\newcommand{\nocontentsline}[3]{}
\newcommand{\tocless}[2]{\bgroup\let\addcontentsline=\nocontentsline#1{#2}\egroup}
