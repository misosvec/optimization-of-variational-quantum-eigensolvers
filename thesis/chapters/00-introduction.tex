\chapter*{Introduction}
\addcontentsline{toc}{chapter}{Introduction}
\markboth{Introduction}{Introduction}

In this day and age we work with so-called NISQ computers, quantum computers that suffers from errors and noise. 


In the first chapter we define some mathematical concepts related to quantum computing and then we explain the very basics of quantum computing. The second chapter describes parametrized quantum circuits, their types and properties. In the third chapter we try to demystify a variational quantum eigensolver algorithm. In the fourth chapter discuss why we have chosen prave the Qiskit library and we explain some terms related to that. \todo{toto je zle aj tak} In the last chapter we present our results and the entire process how we get to them. 


% Put your introduction here. By the way, you can cite articles~\cite{moore_hard_tiling}, books~\cite{golomb_polyomino_book}, websites~\cite{cadical_github}, manuals~\cite{sage}, or other resources. You can reference other chapters, sections, images, etc.\ using the \texttt{cleveref} package like this:~\cref{ch:preliminaries}.