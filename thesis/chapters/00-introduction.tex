\chapter*{Introduction}
\addcontentsline{toc}{chapter}{Introduction}
\markboth{Introduction}{Introduction}

In contrast to classical computers, quantum computers are computers that abide by the laws of quantum mechanics, which describes the behavior of particles. Quantum computers promise to solve problems that are intractable for classical computers~\cite{peruzzo} and so make noticeable advancements in many areas. Nevertheless, this is a matter of the future. In this day and age, we work with so-called noisy intermediate-scale quantum (NISQ) computers~\cite{nisq}. NISQ computers are not that big to tackle difficult instances of problems and they suffer from errors, noise, and decoherence.

In this thesis, we deal with a variational quantum eigensolver (VQE) algorithm~\cite{peruzzo,vqe_method}. The VQE is a hybrid quantum algorithm that combines quantum and classical computing. The quantum part aims to find the energy of a given state, given the Hamiltonian (a matrix that describes a quantum system) and a classical computer runs an optimization algorithm that attempts to find the best parameters for a parametrized quantum circuit, also called ansatz, that defines the state. This algorithm has many applications but the most prominent one is finding a ground state energy (a state where electrons are closest to the nucleus of an atom) of a molecule.

More specifically, this thesis focuses on the performance of the VQE algorithm. We are particularly interested in how various ansatzes and optimization algorithms can affect the performance of the VQE. Hence, we benchmarked the performance of 18 ansatzes and 15 optimizers. For this purpose, we chose a 4-qubit (quantum bit) representation of a hydrogen molecule. We used VQE from the Qiskit~\cite{qiskit} library and we ran VQE on a classical computer with a quantum simulator that does not incorporate noise and finite statistics.

The first four chapters primarily cover the theoretical basics of quantum computing and the VQE algorithm. In the fifth chapter, we present our observed results from the conducted experiments.