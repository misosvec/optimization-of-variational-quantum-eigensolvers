\chapter*{Introduction}
\addcontentsline{toc}{chapter}{Introduction}
\markboth{Introduction}{Introduction}

In contrast to classical computers, quantum computers are computers that abide by laws of quantum mechanics \ques{or physics instead of mechanics?}, which describes the behavior of particles that we cannot see by the naked eye. Quantum computers promise to solve problems that are intractable for classical computers \cite{Peruzzo} and so make noticeable advancements in many areas. Nevertheless, this is a matter of the future. In this day and age, we work with so-called noisy intermediate-scale quantum (NISQ) computers. NISQ computers are not that big to tackle difficult instances of problems and they suffer from errors, noise, and decoherence.

In this thesis, we deal with a variational quantum eigensolver (VQE) algorithm. VQE is a hybrid quantum algorithm that combines quantum and classical computing. The quantum part aims to solve for the lowest eigenvalue \todo{or expectation value?, study it more} of the Hamiltonian (a matrix that describes the quantum system) and a classical computer runs an optimization algorithm that attempts to find the best parameters for a parametrized quantum circuit, also called ansatz. This algorithm has many applications, but the most prominent one is finding a ground state energy (a state where electrons are closest to the nucleus of an atom) of a molecule.

The goal of this thesis is to find interesting relationships from data produced by many VQE runs. We primarily focused on a relationship between various ansatzes and classical optimizers. \ques{this text causes discrepancy with the thesis assignment, not sure how to approach it...} We have created a benchmark \ques{I'm not sure whether we want to use the word benchmark} that assesses the performance of 18 parametrized quantum circuits and 15 optimizers. For this purpose, we have chosen a 4-qubit (quantum bit) representation of a hydrogen molecule. We used VQE from the Qiskit library and we ran VQE solely on a classical computer with a quantum simulator that does not incorporate noise.

% instead of solely I used entirely

The first four chapters primarily cover the theoretical basics of quantum computing and the VQE algorithm. In the fifth chapter, we present our observed results from the conducted experiments.

% In the first chapter, we define some mathematical concepts related to quantum computing and then we explain the very basics of quantum computing. The second chapter describes parametrized quantum circuits, their types and properties. In the third chapter, we try to demystify a variational quantum eigensolver algorithm. In the fourth chapter discuss why we have chosen prave the Qiskit library and explain some terms related to that. \todo{toto je zle aj tak} In the last chapter we present our results and the entire process of how we get to them. 


% Put your introduction here. By the way, you can cite articles~\cite{moore_hard_tiling}, books~\cite{golomb_polyomino_book}, websites~\cite{cadical_github}, manuals~\cite{sage}, or other resources. You can reference other chapters, sections, images, etc.\ using the \texttt{cleveref} package like this:~\cref{ch:preliminaries}.