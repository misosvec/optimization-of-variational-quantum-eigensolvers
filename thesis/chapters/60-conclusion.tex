\chapter*{Conclusion}
\addcontentsline{toc}{chapter}{Conclusion}
\markboth{Conclusion}{Conclusion}


In this bachelor's thesis, we focused on a variational quantum eigensolver algorithm that was used to find the ground state energy of a hydrogen molecule. However, the performance of VQE depends heavily on a choice of ansatz (parametrized quantum circuit) and optimization algorithm. We created a benchmark whose task is to compare the performance of various ansatzes and optimizers.

The main outcome of our work is that gradient-based optimizers work better than gradient-free optimizers. The gradient-based optimizers either reached very good results or completely failed regardless of the chosen ansatz. We think that such a difference in performance was caused primarily by a fixed number of cost function evaluations. We believe that the performance of those optimizers can be improved by allowing more iterations and tuning the hyperparameters. The results of gradient-free optimizers are more varied. It seems that gradient-free optimizers can get the work done but they are more dependent on the chosen ansatz. The interesting is fact that gradient-free optimizers reach better results with 2-layers, which can mean that the performance of the gradient-free optimizers can decrease with an increasing number of cost function parameters. Overall, we think that a choice of optimizer is more important the the choice of ansatz, at least in ideal conditions where noise and errors are not present.

There are many ways how advance our work. It would be interesting to try our benchmark with a broader set of problems than just a hydrogen molecule and see whether our results can be translated there. Moreover, we could hand-pick some optimizers, delve deeper into their properties and tweak their hyperparameters to get the most out of it. \todo{this needs to be extended and written better}