\chapter*{Conclusion}
\addcontentsline{toc}{chapter}{Conclusion}
\markboth{Conclusion}{Conclusion}

In this bachelor's thesis, we focused on a variational quantum eigensolver algorithm that was used to find the ground state energy of a hydrogen molecule. However, the performance of the VQE varies based on a chosen ansatz and optimization algorithm. We evaluated the performance of various ansatzes and optimization algorithms under ideal conditions.

The main outcome of our work is that gradient-based optimizers have a higher chance of yielding correct result than gradient-free optimizers. The gradient-based optimizers reached either very good results or completely failed regardless of the chosen ansatz. We assume that this notable difference in performance was primarily due to a fixed number of cost function evaluations. There could be a chance to enhance the performance of bad-performing optimizers, at least the gradient-based, by increasing the number of iterations and adjusting their parameters. On the other hand, gradient-free optimizers can operate faster, however, at a cost of a lower probability of reaching chemical precision. Also, the choice of ansatz seems to have a greater impact than in gradient-based optimizers. Another intriguing finding is that gradient-free optimizers reach better results with 2-layer ansatzes and gradient-based with 3-layer ansatzes, 1-layer ansatzes do not work at all. \textit{SLSQP (Sequential Least SQuares Programming)} appears to be the best optimizer in terms of speed and probability of reaching a chemical precision. Overall, we think that a choice of optimizer is more important than a choice of ansatz, at least in ideal conditions where noise and errors are not present.

There are many ways how advance our work. It would be interesting to try our benchmark with a broader set of problems or on a problem of a larger scale than just a hydrogen molecule and see whether our results can be transferred there. Moreover, we could hand-pick some optimizers, thoroughly investigate their characteristics, and adjust their parameters to maximize their effectiveness. Furthermore, experimenting with this benchmark on a simulator incorporating noise and errors would provide a more realistic assessment of the optimizers' capabilities. Another option is to execute this benchmark on a real quantum computer, however, this would be a costly approach.