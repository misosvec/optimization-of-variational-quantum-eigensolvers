\chapter{Preliminaries}\label{ch:preliminaries}
In this chapter, we will introduce all the necessary mathematical and quantum concepts that will be used in the rest of the thesis. It is important to understand these concepts before we move on further. We will start with the basic linear algebra concepts and then we will move on to the quantum computing concepts.
\section{Mathematics of quantum computing}
In the case of standard computers, we use boolean algebra. Quantum computing leverages the power of linear algebra. In this section, we will build upon generic linear algebra concepts and introduce some of the concepts that are specific to physics.

\subsection{Bra-ket notation}
Bra-ket notation, also known as Dirac notation plays an important role in quantum mechanics. It is a notation for vectors and matrices and is used to describe a quantum state. 

The main advantage of this notation is that enables us to easily write vector operations such as inner product, outer product, tensor product, etc.

\begin{figure}[h!]
    \begin{minipage}[b]{.5\textwidth}
    \centering
    $$\bra{\alpha} = \begin{pmatrix}
        a_1 \\
        a_2 \\
        \vdots \\
        a_n
    \end{pmatrix}$$
    \caption*{Bra}
    \end{minipage}
    \hfill
    \begin{minipage}[b]{.5\textwidth}
    \centering
    $$\ket{\beta} = \begin{pmatrix}
        b_1 & b_2 & \hdots & b_n
    \end{pmatrix}
    $$
    \caption*{Ket}
    \end{minipage}
\end{figure}

In simple words, a bra is a column vector and a ket is a row vector. Thanks to this notation we can easily write the inner product as $\braket{\alpha}{\beta}$, the outer product as $\ket{\alpha}\bra{\beta}$ and the Kronecker product as $\ket{\alpha}\otimes\ket{\beta}$. The Kronecker product is also one of the operations that is heavily used in quantum computing. We will discuss it in more detail in the following section.
\\ \todo{fix alignment of ket vector}
\\ \todo{consider adding a simple explanation of inner and outer product}
\\ \todo{inner product vs dot product}
\subsection{Kronecker product}
Besides the standard inner and outer product operations, the Kronecker product is another operation that is heavily used in quantum computing. It is a binary operation that combines two matrices into one larger matrix. The Kronecker product is denoted by $\otimes$ and is defined as follows:
$$A \otimes B = \begin{pmatrix}
    a_{11}B & a_{12}B & \hdots & a_{1n}B \\
    a_{21}B & a_{22}B & \hdots & a_{2n}B \\
    \vdots & \vdots & \ddots & \vdots \\
    a_{m1}B & a_{m2}B & \hdots & a_{mn}B
\end{pmatrix}$$

Essentially, we are multiplying each element of the first matrix by the second matrix. The result is a larger block matrix. Kronecker product is a specialization of the tensor product. Sometimes are these operations used interchangeably since they use the same notation $\otimes$.

\subsection{Hilbert space}
Mostly, we are used to working with finite-dimensional vector spaces. Recall that the dimension of vector space is the number of vectors required to form a basis.

In quantum computing, we work with infinite-dimensional vector spaces over $\mathbb{C}$. Virtually, Hilbert space is a standard vector space over $\mathbb{C}$, but in addition to that, it is equipped with a complete inner product.

By complete, we mean that every infinite series of vectors in Hilbert space converges to another vector in Hilbert space. $$\sum_{i=0}^{\infty}\vert a_i \vert < \infty$$

\section{Introduction to quantum computing}
\textcolor{red}{TODO: consider adding motivation, maybe Moors law}

\subsection{Qubit}
\textcolor{red}{TODO: global and relative phase may be also helpful}

\subsection{Superposition}
\subsection{Quantum entanglement}
\subsection{Quantum gates}
\todo{consider the definition of unitary and hermitian matrices}
\textcolor{red}{TODO: reversibility, logical AND does not have an inverse function...}