\chapter{Ansatz optimization}\label{ch:ansatz}
In this chapter, we take a closer look at the ansatz optimization problem. We start by introducing the ansatz optimization problem. We then discuss the different types of ansatzes. Finally, we discuss the different optimization methods.
\section{Our problem}
The goal of this thesis is to optimize the ansatz (parametrized quantum circuit) for a given quantum problem. We want to find the simplest ansatz in terms of circuit depth and number of used gates. Lots of used gates and high circuit depth can bring a higher amount of noise and errors. Input for the VQE algorithm is ansatz and a Hamiltonian (a function that describes the behavior of a physical system). The VQE is a hybrid algorithm, but this thesis is more focused on that quantum part.

We do not aim to create ansatz for any general problem. We will be working with a specific problem. We are considering a molecule of lithium hydride (LiH). To find the ground state energy of this molecule, we will use 8 qubits. We will map the Hamiltonian of that molecule to 8 qubits. We will then use the VQE algorithm to find the ground state energy of the molecule. The VQE algorithm requires an ansatz. We will optimize the ansatz for this specific problem. 

The goal of the given homework was to compare our work with different solutions that already exist. To be honest, I had difficulties finding a similar work. We are considering a specific molecule with a specific amount of qubits, therefore, it is not easy to find similar work.

I found a few related articles \cite{linkedin, medium, innostation}. They just show how things work, but they do not address ansatz optimization. Despite that, they can be a good resource for learning purposes.

Other than that, there may be some other scientific articles that are way above our thesis. Also, at this point, I am not eligible to understand that. It requires extensive knowledge of physics.

\section{Quantum circuits}
\section{Parametrized quantum circuits}
\section{Hardware efficient ansatz}
\section{Linear entanglement}
\section{Cyclic entanglement}
\section{Full entanglement}


