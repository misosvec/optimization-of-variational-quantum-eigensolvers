\chapter{Variational quantum eigensolver}\label{ch:vqe}
At first glance, the term variational quantum eigensolver may seem complicated and it does not say anything to people outside of quantum physics. Thus, the goal of this chapter is to make clear what the variational quantum eigensolver actually is and why it is termed as it is. For the rest of this thesis, we will use the abbreviation VQE.\\
\textcolor{red}{TODO: consider including the following points:\\ eigenvalue, eigenvectors \\ why do we compute that on a quantum computer when a standard computer can also compute eigenvalues and eigenvectors in a polynomial time? }

\section{Variational principle}
\section{Eigensolver}
\section{Hamiltonian}
\textcolor{red}{TODO: different forms, matrix, etc...}

\section{Ground state energy}